\documentclass[conference]{IEEEtran}
\IEEEoverridecommandlockouts
% The preceding line is only needed to identify funding in the first footnote. If that is unneeded, please comment it out.
\usepackage{gensymb}
\usepackage[spanish]{babel}
\usepackage[utf8]{inputenc}
\usepackage{cite}
\usepackage{amsmath,amssymb,amsfonts}
\usepackage{algorithmic}
\usepackage{graphicx}
\usepackage{textcomp}
\usepackage{xcolor}
\def\BibTeX{{\rm B\kern-.05em{\sc i\kern-.025em b}\kern-.08em
    T\kern-.1667em\lower.7ex\hbox{E}\kern-.125emX}}
\begin{document}

\title{Proyecto de investigación - Hilos.
\\
{\footnotesize\textsuperscript{Informática II}
}
}



\author{\IEEEauthorblockN{1\textsuperscript{st} Paubla Andrea Rivera Anaya}
\IEEEauthorblockA{\textit{Facultad de Ingeniería} \\
\textit{Universidad de Antioquia}\\
Medellín, Colombia \\
paubla.rivera@udea.edu.co}




}

\maketitle

\begin{abstract}

This report talks about threads, a feature that allows an application to multitask concurrently. Using these allows you to simplify the design of an application that must carry out different functions simultaneously. \\


\end{abstract}

\begin{IEEEkeywords}

.
\end{IEEEkeywords}
\\
\\

\section{INTRODUCCIÓN}
\\
\\

Se ha dado gran importancia al desarrollo de algoritmos y programas que permitan resolver distintos problemas en cualquier campo que se requiera, por esta razón la implementación de hilos en un sistema operativo es indispensable, ya que estos aumentan la eficiencia de la comunicación entre programas en ejecución. En la mayoría de los sistemas la comunicación entre procesos debe intervenir el núcleo para ofrecer protección de los recursos y realizar la comunicación misma. En cambio, entre hilos pueden comunicarse entre si sin la invocación al núcleo. Por lo tanto, si hay una aplicación que debe implementarse como un conjunto de unidades de ejecución relacionadas, es más eficiente hacerlo con una colección de hilos que con una colección de procesos separados.

\\
\section{OBJETIVOS}
\\
* Reconocer la utilidad de la implementación de hilos de ejecución en un programa.
\\\\
* Identificar tipos de hilos.
\\\\
* Aplicar hilos en usando el lenguaje C++.
\\\\


 

\section{MARCO TEÓRICO}
\\\\

HISTORIA
Los hilos son una aparición temprana bajo el nombre de "tareas" en OS / 360 multiprogramación con un número variable de Tareas (MVT) en 1967. Saltzer (1966) Créditos Victor A. Vyssotsky con el término "hilo". Los programadores de proceso de muchos sistemas operativos modernos soportan directamente tanto tiempo en rodajas y multiprocesador roscado, y el núcleo del sistema operativo permite a los programadores para manipular los hilos mediante la exposición de la funcionalidad requerida por el sistema de llamada de interfaz. Algunas implementaciones de roscado se denominan hilos del núcleo , mientras que los procesos de peso ligero (LWP) son un tipo específico de hilo del núcleo que comparten el mismo estado y la información. Además, los programas pueden tener hilos de espacio de usuario cuando roscar con temporizadores, señales, u otros métodos para interrumpir su propia ejecución, realizando una especie de ad hoc tiempo compartido.
Hilo (informática). (2020, 19 de junio). Wikipedia, La enciclopedia libre. Fecha de consulta: 20:56, julio 12, 2020 desde https://es.wikipedia.org/w/index.php?title=Hilo_(inform%C3%A1tica)&oldid=127057768.

CONTENIDO
Un procesador se encarga de llevar a cabo y ejecutar las instrucciones de los programas que están cargados en la memoria RAM de nuestro ordenador. Por él pasan prácticamente todas las instrucciones que son necesarias para realizar las típicas tareas en nuestro PC, navegar, escribir, ver fotos, etc.
Pues este pequeño chip alberga en su interior distintos módulos que podemos llamar núcleos, además de otros elementos que ahora no nos interesan. Los procesadores de hace unos años tenían uno solo de estos núcleos, y eran capaces de procesar una instrucción por cada ciclo. Estos ciclos se miden en Megahertzios (MHz), mientras más MHz, mas instrucciones podremos hacer cada segundo.

Ahora no solo tenemos un núcleo, sino varios. Cada núcleo representa un subprocesador, es decir, que cada uno de estos subprocesadores ejecutará una de estas instrucciones, pudiendo así ejecutar varias de ellas en cada ciclo de reloj con una CPU de varios núcleos. Si tenemos un procesador de 4 núcleos, podremos ejecutar 4 instrucciones de forma simultánea en lugar de una sola. Entonces, la mejora de rendimiento se cuadriplica. Si tenemos 6, pues 6 instrucciones al mismo tiempo. De esta forma es como los procesadores actuales son muchísimo más potentes que los antiguos.
Podemos definir un hilo de procesamiento como el flujo de control de datos de un programa. Es un medio que permite administrar las tareas de un procesador y de sus diferentes núcleos de una forma más eficiente. Gracias a los hilos, las unidades mínimas de asignación, que son las tareas o procesos de un programa, pueden dividirse en trozos para así optimizar los tiempos de espera de cada instrucción en la cola del proceso. Estos trozos se llaman subprocesos o threads.

 
Publicado en: https://www.profesionalreview.com/2019/04/03/que-son-los-hilos-de-un-procesador/



Un hilo de ejecución, en los sistemas operativos, es similar a un proceso en que ambos representan una secuencia simple de instrucciones ejecutada en paralelo con otras secuencias. Los hilos permiten dividir un programa en dos o más tareas que corren simultáneamente, por medio de la multiprogramación. En realidad, este método permite incrementar el rendimiento de un procesador de manera considerable. En todos los  sistemas de hoy en día los hilos son utilizados para simplificar la estructura de un programa que lleva a cabo diferentes funciones.
Todos los hilos de un proceso comparten los recursos del proceso. Residen en el mismo espacio de direcciones y tienen acceso a los mismos datos. Cuando un hilo modifica un dato en la memoria, los otros hilos utilizan el resultado cuando acceden al dato. Cada hilo tiene su propio estado, su propio contador, su propia pila y su propia  copia de los registros de la CPU. Los valores comunes se guardan en el bloque de control de proceso (PCB), y los valores propios en el bloque de control de hilo (TCB).
Muchos lenguajes de programación (como Java), y otros entornos de desarrollo soportan los llamados hilos o hebras (en inglés, threads).
Un ejemplo de la utilización de hilos es tener un hilo atento a la interfaz gráfica (iconos, botones, ventanas), mientras otro hilo hace una larga operación internamente.
De esta manera el programa responde más ágilmente a la interacción con el usuario.
Los hilos se distinguen de los tradicionales procesos en que los procesos son generalmente independientes, llevan bastante información de estados, e interactúan sólo a través de mecanismos de comunicación dados por el sistema. Por otra parte, muchos hilos generalmente comparten otros recursos directamente. En los sistemas operativos que proveen facilidades para los hilos, es más rápido cambiar de un hilo a otro dentro del mismo proceso, que cambiar de un proceso a otro. Este fenómeno se debe a que los hilos comparten datos y espacios de direcciones, mientras que los procesos al ser independientes no lo hacen. Al cambiar de un proceso a otro el sistema operativo (mediante el dispacher) genera lo que se conoce como overhead, que es tiempo desperdiciado por el procesador para realizar un cambio de modo (mode switch), en este caso pasar del estado de Runnig al estado de Waiting o Bloqueado y colocar el nuevo proceso en Running. En los hilos como pertenecen a un mismo proceso al realizar un cambio de hilo este overhead es casi despreciable.

Al igual que los procesos, los hilos poseen un estado de ejecución y pueden sincronizarse entre ellos para evitar problemas de compartimiento de recursos. Generalmente, cada hilo tiene especificada una tarea específica y determinada, como forma de aumentar la eficiencia del uso del procesador. Los principales estados de ejecución de los hilos son: Ejecución, Listo y Bloqueado. No tiene sentido asociar estados de suspensión de hilos ya que es un concepto de proceso. En todo caso, si un proceso está expulsado de la memoria principal (ram), todos sus hilos deberán estarlo ya que todos comparten el espacio de direcciones del proceso. Las transiciones entre estados más comunes son las siguientes: • Creación: Cuando se crea un proceso se crea un hilo para ese proceso. Luego, este hilo puede crear otros hilos dentro del mismo proceso. El hilo tendrá su propio contexto y su propio espacio de pila, y pasara a la cola de listos. • Bloqueo: Cuando un hilo necesita esperar por un suceso, se bloquea (salvando sus registros). Ahora el procesador podrá pasar a ejecutar otro hilo que este en la cola de Listos mientras el anterior permanece bloqueado. • Desbloqueo: Cuando el suceso por el que el hilo se bloqueo se produce, el mismo pasa a la cola de Listos. • Terminación: Cuando un hilo finaliza se liberan tanto su contexto como sus pilas
http://bibing.us.es/proyectos/abreproy/11320/fichero/Capitulos%252F13.pdf

¿QUE TIPO DE HILOS EXISTEN?
Los Sistemas Operativos generalmente implementan hilos de dos maneras:
 • Multihilo apropiativo: Permite al sistema operativo determinar cuándo debe haber un cambio de contexto. La desventaja de esto es que el sistema puede hacer un cambio de contexto en un momento inadecuado, causando un fenómeno conocido como inversión de prioridades y otros problemas. 
• Multihilo cooperativo: Depende del mismo hilo abandonar el control cuando llega a un punto de detención, lo cual puede traer problemas cuando el hilo espera la disponibilidad de un recurso.
¿CÓMO SE IMPLEMENTAN LOS HILOS POR SOFTWARE? DEBE QUEDAR CLARO SI EL LENGUAJE DE PROGRAMACIÓN IMPORTA Y SI EL HARDWARE USADO AFECTA.
Todos los hilos comparten el mismo espacio de direcciones y otros recursos como pueden ser archivos abiertos. Cualquier modificación de un recurso desde un hilo afecta al entorno del resto de los hilos del mismo proceso. Por lo tanto, es necesario sincronizar la actividad de los distintos hilos para que no interfieran unos con otros o corrompan estructuras de datos.

Hay dos grandes categorías en la implementación de hilos:

•	Hilos a nivel de usuario.
•	Hilos a nivel de núcleo.
También conocidos como ULT (user level thread) y KLT (kernel level thread).

Hilos a nivel de usuario (ULT)
En una aplicación ULT pura, todo el trabajo de gestión de hilos lo realiza la aplicación, y el núcleo o kernel no es consciente de la existencia de hilos. Es posible programar una aplicación como multihilo mediante una biblioteca de hilos. La misma contiene el código para crear y destruir hilos, intercambiar mensajes y datos entre hilos, para planificar la ejecución de hilos y para salvar y restaurar el contexto de los hilos.
Todas las operaciones descritas se llevan a cabo en el espacio de usuario de un mismo proceso. El núcleo continua planificando el proceso como una unidad y asignándole un único estado (Listo, bloqueado, etc.).
Hilos a nivel de núcleo (KLT)
En una aplicación KLT pura, todo el trabajo de gestión de hilos lo realiza el núcleo. En el área de la aplicación no hay código de gestión de hilos, únicamente un API (interfaz de programas de aplicación) para la gestión de hilos en el núcleo. Windows 2000, Linux y OS/2 utilizan este método. Linux utiliza un método muy particular en el que no hace diferencia entre procesos e hilos. Para Linux, si varios procesos creados con la llamada al sistema "clone" comparten el mismo espacio de direcciones virtuales, el sistema operativo los trata como hilos, y lógicamente son manejados por el núcleo.
https://es.wikipedia.org/wiki/Hilo_%28inform%C3%A1tica%29#Implementaciones




\\\\

\section{EJEMPLO DE APLICACIÓN}






\section{ANÁLISIS}
\begin{itemize}
\\







\end{itemize}
\\\\

\begin{thebibliography}{00}


\bibitem{b1}  

\bibitem{b2} 

\bibitem{b3}  

\bibitem{b4}  

\bibitem{b5}  

\bibitem{b6} 

\bibitem{b7} 

\bibitem{b8}  


\end{thebibliography}


\end{document}
